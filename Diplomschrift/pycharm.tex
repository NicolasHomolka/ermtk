\section{Pycharm als IDE}
\fib{}
\subsection{Allgemeines}

\subsubsection{Version}

\noindent
PyCharm ist eine IDE vom Unternehmen JetBrains, die im Juli 2010 erschienen ist und viele Eigenschaften sowie Features mit den anderen IDEs von JetBrains teilt. Dabei hat JetBrains wie der Name PyCharm schon andeutet, die IDE explizit für die Programmiersprache Python entwickelt. Die aktuellste Version ist die Version 2018.3.5, die am 27. Februar 2019 erschienen ist. Jedoch ist die neue Version 2019.1 schon in Entwicklung und wird noch im zweiten Quartal von 2019 verfügbar sein. 
In der Regel werden jedes Jahr ca. drei neue Versionen veröffentlicht, jedoch handelt es sich dabei eher um kleinere Updates oder Hotfixes und nicht um große Erneuerungen.
\footcite{noauthor_pycharm:_nodate}

\subsubsection{Editionen}

\noindent
Außerdem gibt es PyCharm in drei verschiedenen Varianten, einmal die PyCharm Community Edition, die PyCharm Professional Edition und die PyCharm Educational Edition.
\footcite{noauthor_previous_nodate}

\noindent
\begin{itemize}
    \item Die Community Edition ist Open-Source und damit gratis für jeden erhältlich. 
    \item Die Professional Edition verfügt über mehr Features, man benötigt für diese Version jedoch eine Lizenz die man kaufen muss.Für dieses Projekt wurde die Professional Edition verwendet, da JetBrains die Lizenzen gratis für Schulen anbietet.
    \item Die dritte Edition ist zum Erlernen von Python gedacht.
\end{itemize}



\subsection{Funktionen}
\subsubsection{Intelligenter Code Editor}

\noindent
PyCharm bietet intelligente Code Vervollständigung, Syntax Hervorhebung, automatische Code Refactorings.
Die IDE erkennt wenn Code-Teile kopiert wurden und macht dem entsprechen Vorschläge für das Refactoring.
\footcite{noauthor_features_nodate}

\subsubsection{Web Development Frameworks}

\noindent
PyCharm unterstützt auch spezifische Web Development Frameworks wie z.B.:

\begin{itemize}
    \item Django
    \item Flask
    \item Google App Engine
    \item web2py
\end{itemize}

\subsubsection{Cross-technology Development}

\noindent
Neben Python unterstützt die IDE auch noch andere Sprachen wie z.B.: JavaScript, TypeScript, SQL und HTML/CSS.

\subsubsection{Built-in Developer Tools}

\noindent
Debuggen, Testen und Profiling ist mit PyCharm durch eine Vielzahl an GUI Elementen einfach und schnell möglich. Automatisches Einsätzen auf einem entfernten Rechner kann einfach eingestellt werden und außerdem bietet die IDE eine schnelle und benutzerfreundliche GUI für Versionsverwaltungsprotokolle wie z.B.: Git, Mercurial und SVN.

\subsection{Vorteile}
\fib{}
\noindent
\begin{itemize}
\item Ein Vorteilen der IDE ist der Support der mittels Forum oder Mail Kontakt direkt mit den Entwicklern erfolgt.
\item Neben dem Support wird bei JetBrains auch die Benutzerfreundlichkeit ihrer IDE. Die einzelnen Funktionen sind übersichtlich von den anderen getrennt und in den Untermenü mit gut erkennbaren Symbolen gekennzeichnet.
\item Der WSL Interpreter war bei dem Projekt eine große Hilfe, da nicht jeder auf dem selben Betriebssystem arbeitete. Dadurch konnte man selbst, wenn auf einem Rechner mit Windows gearbeitet wurde, in einer Linux Umgebung testen.
\item Durch die große Community von PyCharm und JetBrains gibt es für jede IDE eine Vielzahl an Plugins, die nicht nur das Programmieren vereinfachen, sondern auch die Versionsverwaltung.
\item Durch die REPL Python Konsole wird dem Entwickler das Testen vereinfacht, da man direkt von der IDE aus das Programm starten, testen und Fehlerbehebung kann. Außerdem wird einem der derzeitige Zustand der Variablen angezeigt.
\end{itemize}
\footcite{noauthor_features_nodate}