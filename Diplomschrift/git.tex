\section{Git}
\fib{}

\noindent
Bei einem Projekt mit diesem Umfang ist es von Vorteil jegliche Art von Versionsverwaltung zu verwenden. Außer Git hätten sich noch andere Versionsverwaltungssysteme angeboten wie z.B.:
\footcite{noauthor_git_2019}
\begin{itemize}
	\item Mercurial
	\item Darcs
	\item Fossil
	\item Bazaarn	
\end{itemize}

\noindent
Jedoch wurde für dieses Projekt gezielt Git verwendet. Git ist unter den Versionverwaltungssystemen das gängigste und deswegen existiert dafür auch der meiste Support. Außerdem verwendet Trac, die Projektorganisations-Web-App die in diesem Projekt verwendet wurde, Git deswegen waren die anderen Möglichkeiten nicht mit Git gleichwertig. 
\\

\noindent
Durch gezieltes anlegen von Branches wurde es ermöglicht jedem Entwickler eine vollkommen von den Anderen unabhängige Entwicklungsumgebung zu bieten. 
Diese sogenannten Feature-Branches ermöglichten es, unabhängig vom derzeitigen Entwicklungsstands des Projekts, es zu jederzeit eine funktionierende Version der Software am Master-Branch vorlag.
\\

\noindent
Die Daten lagen zu jeder Zeit auf dem Server, den der Auftraggeber bereitgestellt hat, vor und ermöglichte dadurch eine einfachen Zugriff. Mit einem unabhängigen Server ist es den Entwicklern leicht gefallen von zu Hause und in der Schule immer am neusten Stand zu sein.
\\