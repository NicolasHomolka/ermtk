 
\section{Fußball}
\pra

\noindent
Für die Erstellung des Datenmodells \textit{Fußball} wurden folgende Annahmen getroffen:

\begin{itemize}
	\item Jede Mannschaft hat einen eindeutigen Namen, ein bestimmtes Gründungsjahr und ist an einer bestimmten Adresse beheimatet.
	\item Zu jeder Mannschaft gehören Fußballspieler.
	\item Ein Spieler kann durch die SVNr identifiziert werden. Weiters hat ein Spieler einen Namen, eine Wohnadresse, ein Geburtsdatum und eine Position, an der er spielt.
	\item Die Mannschaften beteiligen sich an Spielen. 
	\item Die Spiele können durch die Adresse des Stadions, dem Tag und der Uhrzeit eindeutig festgelegt werden.
	\item Pro Spiel werden die beteiligten Mannschaften sowie der Schiedsrichter und das Ergebnis gespeichert.
	\item Falls das Spiel zu einem Turnier gehört, soll diese Information ebenfalls gespeichert werden.
	\item Die Anzahl der Tore, die in einem Spiel geschossen wurden, sollen gespeichert werden.
	\item Ein Schiedsrichter verfügt über die gleichen Daten wie ein Spieler ausser dass er keine Spielposition hat. Dafür wird bei dem Schiedsrichter das Datum der Schiedsrichterprüfung und die Berechtigungsklasse gespeichert.
	\item Jedes Turnier hat eine eindeutige Nummer, einen Namen, ein Beginn- und Enddatum und die beteiligten Mannschaften gespeichert.
\end{itemize}

\noindent
Anhand all dieser Annahmen entsteht folgendes Datenmodell:\pra
\begin{verbatim}
<erm version="0.2">

<!-- Front Matter -->

<title name="Fussball"/>
<title name="soccer" lang="en"/>

<!-- Entity-Types -->

<ent name="mannschaft">
    <attr name="name" prime="true"/>
    <attr name="gründungsjahr"/>
    <attr name="adresse"/>
</ent>

<ent name="person">
    <attr name="svnr" prime="true"/>
    <attr name="name"/>
    <attr name="wohnadresse"/>
    <attr name="geburtsdatum"/>
</ent>

<ent name="spieler">
    <attr name="spielposition"/>
</ent>

<ent name="spiel">
   <attr name="spielort" prime="true"/>
   <attr name="datum"    prime="true"/>
   <attr name="mannschaft_heim"/>
   <attr name="mannschaft_ausw"/>
   <attr name="schiedsrichter"/>
   <attr name="ergebnis"/>
</ent>

<ent name="turnier">
   <attr name="nummer" prime="true"/>
   <attr name="name"   prime="true"/>
   <attr name="beginndatum"/>
   <attr name="enddatum"/>
   <attr name="mannschaften"/>
</ent>
 
<ent name="schiedsrichter">
   <attr name="datum_prüfung"/>
   <attr name="berechtigungsklasse"/>
</ent>

<ent name="tore">
   <attr name="anzahl_tore"/>
</ent>

\end{verbatim}

\noindent
Die \textit{Entities} stehen wie folgt in Beziehung miteinander:
\pra
\\

\begin{verbatim}
<rel to="ist ein">
    <super ref="person" total="false" disjoint="true"/>
        <sub ref="spieler"/>
        <sub ref="schiedsrichter"/>
</rel>

<rel to="spielt bei">
    <part ref="spieler" min="1" max="1"/>
    <part ref="mannschaft" min="1" max="n"/>
</rel>

<rel to="spielt mit bei">
    <part ref="mannschaft" min="1" max="n"/>
    <part ref="spiel" min="1" max="n"/>
</rel>

<rel to="gehört zu">
    <part ref="spiel" min="0" max="1"/>
    <part ref="turnier" min="1" max="n"/>
</rel>

<rel to="pfeift bei">
    <part ref="schiedsrichter" min="1" max="1"/>
    <part ref="spiel" min="1" max="1"/>
</rel>

<rel to="hat geschossen">
    <part ref="tore" min="0" max="n"/>
    <part ref="spieler" min="0" max="n"/>
</rel>

<rel to="wurden geschossen">
    <part ref="tore" min="0" max="n"/>
    <part ref="spiel" min="0" max="n"/>
</rel>
</erm>

\end{verbatim}
\pra