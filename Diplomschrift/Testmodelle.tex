\chapter{Testmodelle}
\label{cha:Testmodelle}
\fib{}
\section{Überblick}

\noindent
Bei dem Projekt gab es 9 Testmodelle die wären:
\begin{itemize}
    \item Fußball
    \item Kinokette
    \item Mondial
    \item Rettungsstelle
    \item Schulungsfirma
    \item Schulinformationssystem
    \item Tankstellenkette
    \item Weingut
\end{itemize}
\noindent

\noindent
Jedes dieser Testmodelle stellt eine Datenbank da. Alle 9 mussten in Form eine XERML-Datei erstellt werden um sie dann mit den vier verschiedenen Programmen zu verarbeiten. Darin sollten alle Variationen vorhanden sein z.B. Vererbungen, Abhängige-Typen usw..

\noindent
Diese XERML-Dateien können mit einem Tool eines anderen Projekts automatisch generiert werden.
Sie bestehn in der Regel aus drei verschiedenen Dateien.
\begin{itemize}
    \item Grunddatei mit der Endung "xerml.xml", in dieser Datei steht der grobe Aufbau des Modells.
    \item Sprachdatei wo verschiedene Übersetzungen enthalten sind mit der Endung "xerml.lo.xml".
    \item Typdatei die die einzelnen Attribute einer Entity beschreibt mit der Endung "xerml.ty.xml".
\end{itemize}


\newpage
\section{Schulinformationssystem}
\prc

Das Testmodell ,,Schulinformationssystem'' soll das Datenmodell für eine HTL abbilden. Dabei waren folgenden Punkte aus dem Skriptum \cite{skriptum} zu berücksichtigen:

\begin{enumerate}
	\item Die abteilungsweise Gliederung einer HTL ist wiederzugegeben.
	\item Jeder Lehrer ist einer Abteilung als Stammabteilung zugeordnet, kann aber auch in Klassen anderer Abteilungen unterrichten. Jede Abteilung wird von einem Lehrer als Abteilungsvorstand geleitet.
	\item Für jede Ausbildungsform der Abteilungen ist im Lehrplan festgehalten, welche Gegenstände in welchen
	Jahrgängen, in welchem Ausmaß (Theorie- und Übungsstunden) unterrichtet werden müssen.
	\item Die Klassen eines Schuljahres werden von den Lehrern in den einzelnen Gegenständen in einem bestimmten
	Stundenausmaß (Theorie- und Übungsstunden) unterrichtet.
	\item Jeder Schüler wird mit einer Semester- und einer Jahresnote pro Klasse und Gegenstand beurteilt. In dem
	System sollen diese Informationen für mehrere Schuljahre festgehalten werden können.
	\item Die Klassenvorstände der verschiedenen Klassen sollen feststellbar sein, ebenso von welchem Schüler welche Funktionen (Klassensprecher, Kassier, etc.) ausgeübt werden oder wurden.
	\item Die Entlohnung der Lehrer erfolgt nicht nach gehaltenen Stunden, sondern nach gehaltenen Werteinheiten:
	jeder Gegenstand ist einer bestimmten Lehrverpflichtungsgruppe (LVG) (I bis VI) zugeordnet. Für jede LVG
	ist ein Faktor (1,167 bis 0,75) festgelegt, der zur Umrechnung von Stunden in Werteinheiten herangezogen
	wird.
	\item Für jeden Schüler ist ein Erziehungsberechtigter verantwortlich (sofern der Schüler nicht eigenberechtigt
	ist). Wenn Geschwister die Schule besuchen, soll dies ebenfalls ermittelt werden können.
	\\
\end{enumerate} 

\noindent
Diese Punkte ließen auf die folgenden Entitytypen schließen:

\begin{multicols}{2}
	\begin{itemize}
		\item Person
		\item Lehrer
		\item Schüler
		\item Abteilung
		\item Gegenstand
		\item Klasse
		\item Lehrverpflichtungsgruppe
		\item Lehrplan
	\end{itemize}
\end{multicols}

\noindent
Um das Datenmodell vollständig abzubilden sind auch Beziehungen zwischen den Entiytypen nötig. Um zum Beispiel den 2. Punkt der Aufgabe zu realisieren, wurden folgende Beziehungstypen definiert:

\begin{lstlisting}[language=XML, caption={XERML-Definition von Punkt 2 der Angabe}]
<!-- Punkt 2-->
<rel to="gehört zu">
   <part ref="Lehrer" min="1" max="1"/>
   <part ref="Abteilung" min="1" max="n"/>
</rel>

<rel to="wird geleitet" from="leitet">
   <part ref="Abteilung" min="1" max="1"/>
   <part ref="Lehrer" min="0" max="1"/>
</rel>
\end{lstlisting}

\noindent
Die komplette Umsetzung des Datenmodells befindet sich im Anhang.

\section{Rettungsstelle}
\fib{}

\noindent
Die Rettungsstelle soll den Aufbau einer echten Rettungsstelle darstellen und dabei veranschaulichen welche Vorgänge bzw. Ressourcen miteinander in Beziehung stehen.
Das Datenmodell der Rettungsstelle besteht aus sieben Entity-Typen:

\begin{itemize}
    \item Einsatz oder auch mission
    \item Fahrt oder auch trip
    \item Person oder auch person
    \item Angestellter oder auch employee
    \item Patient oder auch patient
    \item Auto oder auch car
    \item Gragage oder auch garage
\end{itemize}

\noindent
Zwischen diesen Entity-Typen existieren sechs Relationen und jede dieser Relationen kann man mit je zwei Sätzen beschreiben.
\begin{itemize}
    \item Ein Einsatz besteht aus keiner oder mehreren Fahrten.\newline
          Eine Fahrt ist Bestandteil genau eines Einsatzes.
    \item Ein Mitarbeiter nimmt an keiner oder mehreren Fahrten teil. \newline
          Bei einer Fahrt ist mindestens ein Mitarbeiter oder mehrere Mitarbeiter beteiligt. 
    \item Ein Patient wird bei genau einem Einsatz gerettet. \newline
          Bei einer Fahrt wird mindestens ein Patient oder mehrere Patienten gerettet. 
    \item Ein Auto wird bei mindestens einer Fahrt oder mehreren Fahrten verwendet. \newline
          Bei einer Fahrt wird genau ein Auto verwendet.
    \item Ein Auto hat mindestens eine Garage oder mehrere Garagen. \newline
          In einer Garage steht genau ein Auto.
    \item Patient ist eine Person mit einer SVN, KVA, Einsatzbeschreibung, Kennzeichen, Einsatznummer. \newline
          Angestellter ist eine Person mit einer Mittarbeiternummer, Rang, Schulung, Kennzeichen, Einsatznummer.
\end{itemize}

\noindent

\newpage
 
\section{Fußball}
\pra

\noindent
Für die Erstellung des Datenmodells \textit{Fußball} wurden folgende Annahmen getroffen:

\begin{itemize}
	\item Jede Mannschaft hat einen eindeutigen Namen, ein bestimmtes Gründungsjahr und ist an einer bestimmten Adresse beheimatet.
	\item Zu jeder Mannschaft gehören Fußballspieler.
	\item Ein Spieler kann durch die SVNr identifiziert werden. Weiters hat ein Spieler einen Namen, eine Wohnadresse, ein Geburtsdatum und eine Position, an der er spielt.
	\item Die Mannschaften beteiligen sich an Spielen. 
	\item Die Spiele können durch die Adresse des Stadions, dem Tag und der Uhrzeit eindeutig festgelegt werden.
	\item Pro Spiel werden die beteiligten Mannschaften sowie der Schiedsrichter und das Ergebnis gespeichert.
	\item Falls das Spiel zu einem Turnier gehört, soll diese Information ebenfalls gespeichert werden.
	\item Die Anzahl der Tore, die in einem Spiel geschossen wurden, sollen gespeichert werden.
	\item Ein Schiedsrichter verfügt über die gleichen Daten wie ein Spieler ausser dass er keine Spielposition hat. Dafür wird bei dem Schiedsrichter das Datum der Schiedsrichterprüfung und die Berechtigungsklasse gespeichert.
	\item Jedes Turnier hat eine eindeutige Nummer, einen Namen, ein Beginn- und Enddatum und die beteiligten Mannschaften gespeichert.
\end{itemize}

\noindent
Anhand all dieser Annahmen entsteht folgendes Datenmodell:\pra
\begin{verbatim}
<erm version="0.2">

<!-- Front Matter -->

<title name="Fussball"/>
<title name="soccer" lang="en"/>

<!-- Entity-Types -->

<ent name="mannschaft">
    <attr name="name" prime="true"/>
    <attr name="gründungsjahr"/>
    <attr name="adresse"/>
</ent>

<ent name="person">
    <attr name="svnr" prime="true"/>
    <attr name="name"/>
    <attr name="wohnadresse"/>
    <attr name="geburtsdatum"/>
</ent>

<ent name="spieler">
    <attr name="spielposition"/>
</ent>

<ent name="spiel">
   <attr name="spielort" prime="true"/>
   <attr name="datum"    prime="true"/>
   <attr name="mannschaft_heim"/>
   <attr name="mannschaft_ausw"/>
   <attr name="schiedsrichter"/>
   <attr name="ergebnis"/>
</ent>

<ent name="turnier">
   <attr name="nummer" prime="true"/>
   <attr name="name"   prime="true"/>
   <attr name="beginndatum"/>
   <attr name="enddatum"/>
   <attr name="mannschaften"/>
</ent>
 
<ent name="schiedsrichter">
   <attr name="datum_prüfung"/>
   <attr name="berechtigungsklasse"/>
</ent>

<ent name="tore">
   <attr name="anzahl_tore"/>
</ent>

\end{verbatim}

\noindent
Die \textit{Entities} stehen wie folgt in Beziehung miteinander:
\pra
\\

\begin{verbatim}
<rel to="ist ein">
    <super ref="person" total="false" disjoint="true"/>
        <sub ref="spieler"/>
        <sub ref="schiedsrichter"/>
</rel>

<rel to="spielt bei">
    <part ref="spieler" min="1" max="1"/>
    <part ref="mannschaft" min="1" max="n"/>
</rel>

<rel to="spielt mit bei">
    <part ref="mannschaft" min="1" max="n"/>
    <part ref="spiel" min="1" max="n"/>
</rel>

<rel to="gehört zu">
    <part ref="spiel" min="0" max="1"/>
    <part ref="turnier" min="1" max="n"/>
</rel>

<rel to="pfeift bei">
    <part ref="schiedsrichter" min="1" max="1"/>
    <part ref="spiel" min="1" max="1"/>
</rel>

<rel to="hat geschossen">
    <part ref="tore" min="0" max="n"/>
    <part ref="spieler" min="0" max="n"/>
</rel>

<rel to="wurden geschossen">
    <part ref="tore" min="0" max="n"/>
    <part ref="spiel" min="0" max="n"/>
</rel>
</erm>

\end{verbatim}
\pra

\section{Weingut}
\hon{}
\noindent
Im Umfang der Diplomarbeit ist die Erstellung einer \textit{XERML}-Testmodells von jedem Diplomanden beinhaltet, um die zu entwickelnde Software mit den Modellen testen zu können.
\\
\noindent
Gefordert ist das Modell eines Weinguts, in dem verschiedene Erzeuger diverse Weine anbieten. Jeder Erzeuger liegt in einem Anbaugebiet und hat eine Lizenz, die es dem Erzeuger ermöglicht eine bestimme Menge Wein zu erzeugen. Die Weine bestehen jeweils zu einem bestimmten Anteil aus verschiedenen Rebsorten.


\subsection{Die Hauptdatei}
\hon{}

\noindent
Bei der Modellierung der Hauptdatei \textit{weingut.xerml.xml} kommt es zu mehreren Varianten der \textit{Beziehungen} oder \textit{Attribute}. Um das \textit{XERML} so realistisch und einfach wie möglich zu halten, wurden folgende Annahmen getroffen:
\begin{itemize}
	\item Der Name des Weins ist der \textit{Primary-Key}, da es in der Realität nicht mehrere Weine mit dem selben Namen gibt. 
	\item Der Name und die Adresse des Erzeugers bilden den \textit{Primary-Key}, da es keine Erzeuger mit demselben Namen und derselben Adresse geben kann.
	\item Der Name und die Region des Anbaugebiets ist der \textit{Primary-Key}, da es nicht mehrere Anbaugebiete mit demselben Namen in derselben Region geben kann.
	\item Der Name einer Rebsorte ist eindeutig und daher der alleinige \textit{Primary-Key}. 
	\item Die Lizenznummer einer Lizenz ist eindeutig und daher der alleinige \textit{Primary-Key}.
	
	\item Ein bestimmter Erzeuger kann mehrere Weine erzeugen und ein bestimmter Wein kann von mehreren Erzeugern erzeugt werden. Ein Erzeuger muss allerdings mindestens einen Wein erzeugen um als Erzeuger zu gelten. Ein bestimmter Wein muss nicht zwingend hergestellt werden.
	\item Ein bestimmter Wein muss mindestens eine Rebsorte enthalten. Eine Rebsorte muss nicht zwingend in einem Wein enthalten sein. Jede Rebsorte ist zu einem gewissen Anteil in \verb|%| in einem Wein enthalten. 
	\item Ein bestimmter Erzeuger besitzt genau eine Lizenz. Eine Lizenz kann allerdings auch von mehreren Erzeugern besessen werden. Sie muss aber mindestens von einem besessen werden. 
	\item Ein bestimmter Erzeuger muss in mindestens einem Anbaugebiet liegen. In einem bestimmten Anbaugebiet muss nicht zwingend ein Erzeuger anbauen. Es können allerdings mehrere darin anbauen.
\end{itemize}
\noindent
\hon{}
\\
\noindent 
Aufgrund dieser Annahmen ergibt sich folgendes \textit{XERML}:


\begin{verbatim}
<erm version="0.2">

    <title name="Weingut"/>

    <ent name="wein">
        <attr name="name" prime="true"/>
        <attr name="farbe"/>
        <attr name="jahrgang"/>
        <attr name="restsüße"/>
        <attr name="erzeuger"/>
    </ent>

    <ent name="erzeuger">
        <attr name="name" prime="true"/>
        <attr name="adresse" prime="true"/>
        <attr name="lizenz"/>
        <attr name="menge"/>
        <attr name="anbaugebiet"/>
    </ent>

    <ent name="anbaugebiet">
        <attr name="name" prime="true"/>
        <attr name="region" prime="true"/>
        <attr name="land"/>
    </ent>

    <ent name="rebsorte">
        <attr name="name" prime="true"/>
        <attr name="farbe"/>
    </ent>

    <ent name="lizenz">
        <attr name="lizenznummer" prime="true"/>
        <attr name="menge"/>
    </ent>

    <rel to="erzeugt">
        <part ref="erzeuger" min="1" max="n"/>
        <part ref="wein" min="0" max="n"/>
    </rel>

    <rel to="beinhaltet">
        <part ref="wein" min="1" max="n"/>
        <part ref="rebsorte" min="0" max="n"/>
        <attr name="anteil"/>
    </rel>

    <rel to="besitzt">
        <part ref="erzeuger" min="1" max="1"/>
        <part ref="lizenz" min="1" max="n"/>
    </rel>

    <rel to="liegt in einem">
        <part ref="erzeuger" min="1" max="n"/>
        <part ref="anbaugebiet" min="0" max="n"/>
    </rel>

</erm>
\end{verbatim}



\subsection{Die Sprachdatei}
\hon{}

\noindent
Die Sprachdatei \textit{weingut.xerml.lo.xml} beinhaltet eine englische Übersetzung der in der deutschen Sprache geschriebenen Hauptdatei.

\subsection{Die Typdatei}
\hon{}

\noindent
Die Typdatei \textit{weingut.xerml.ty.xml} beinhaltet die Datentypen aller \textit{Attribute}. Die \textit{Attribute} Jahrgang, Lizenznummer und Lizenz haben den Datentyp \verb|Integer|. Die \textit{Attribute} Restsüße, Menge und Anteil haben den Datentypen \verb|float|. Die anderen \textit{Attribute} besitzt den Datentyp \verb|char|.

