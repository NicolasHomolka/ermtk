\section{Schulinformationssystem}
\prc

Das Testmodell ,,Schulinformationssystem'' soll das Datenmodell für eine HTL abbilden. Dabei waren folgenden Punkte aus dem Skriptum \cite{skriptum} zu berücksichtigen:

\begin{enumerate}
	\item Die abteilungsweise Gliederung einer HTL ist wiederzugegeben.
	\item Jeder Lehrer ist einer Abteilung als Stammabteilung zugeordnet, kann aber auch in Klassen anderer Abteilungen unterrichten. Jede Abteilung wird von einem Lehrer als Abteilungsvorstand geleitet.
	\item Für jede Ausbildungsform der Abteilungen ist im Lehrplan festgehalten, welche Gegenstände in welchen
	Jahrgängen, in welchem Ausmaß (Theorie- und Übungsstunden) unterrichtet werden müssen.
	\item Die Klassen eines Schuljahres werden von den Lehrern in den einzelnen Gegenständen in einem bestimmten
	Stundenausmaß (Theorie- und Übungsstunden) unterrichtet.
	\item Jeder Schüler wird mit einer Semester- und einer Jahresnote pro Klasse und Gegenstand beurteilt. In dem
	System sollen diese Informationen für mehrere Schuljahre festgehalten werden können.
	\item Die Klassenvorstände der verschiedenen Klassen sollen feststellbar sein, ebenso von welchem Schüler welche Funktionen (Klassensprecher, Kassier, etc.) ausgeübt werden oder wurden.
	\item Die Entlohnung der Lehrer erfolgt nicht nach gehaltenen Stunden, sondern nach gehaltenen Werteinheiten:
	jeder Gegenstand ist einer bestimmten Lehrverpflichtungsgruppe (LVG) (I bis VI) zugeordnet. Für jede LVG
	ist ein Faktor (1,167 bis 0,75) festgelegt, der zur Umrechnung von Stunden in Werteinheiten herangezogen
	wird.
	\item Für jeden Schüler ist ein Erziehungsberechtigter verantwortlich (sofern der Schüler nicht eigenberechtigt
	ist). Wenn Geschwister die Schule besuchen, soll dies ebenfalls ermittelt werden können.
	\\
\end{enumerate} 

\noindent
Diese Punkte ließen auf die folgenden Entitytypen schließen:

\begin{multicols}{2}
	\begin{itemize}
		\item Person
		\item Lehrer
		\item Schüler
		\item Abteilung
		\item Gegenstand
		\item Klasse
		\item Lehrverpflichtungsgruppe
		\item Lehrplan
	\end{itemize}
\end{multicols}

\noindent
Um das Datenmodell vollständig abzubilden sind auch Beziehungen zwischen den Entiytypen nötig. Um zum Beispiel den 2. Punkt der Aufgabe zu realisieren, wurden folgende Beziehungstypen definiert:

\begin{lstlisting}[language=XML, caption={XERML-Definition von Punkt 2 der Angabe}]
<!-- Punkt 2-->
<rel to="gehört zu">
   <part ref="Lehrer" min="1" max="1"/>
   <part ref="Abteilung" min="1" max="n"/>
</rel>

<rel to="wird geleitet" from="leitet">
   <part ref="Abteilung" min="1" max="1"/>
   <part ref="Lehrer" min="0" max="1"/>
</rel>
\end{lstlisting}

\noindent
Die komplette Umsetzung des Datenmodells befindet sich im Anhang.