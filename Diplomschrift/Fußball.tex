\section{Datenmodell Fußball}
\pra

\noindent
Für die Erstellung des Datenmodells \textit{Fußball} wurden folgende Annahmen getroffen:

\begin{itemize}
	\item Jede Mannschaft hat einen eindeutigen Namen, ein bestimmtes Gründungsjahr und an einer bestimmten Adresse beheimatet.
	\item Zu jeder Mannschaft gehören Fußballspieler
	\item Ein Spieler kann durch die SVNr identifiziert werden. Weiters hat ein Spieler einen Namen, eine Wohnadresse, ein Geburtsdatum und eine Position, an der er spielt.
	\item Die Mannschaften beteiligen sich an Spielen. 
	\item Die Spiele können durch die Adresse des Stadions, dem Tag und der Uhrzeit eindeutig festgelegt werden.
	\item Pro Spiel werden die beteiligten Mannschaften sowie Schiedsrichter und das Ergebnis gespeichert werden.
	\item Falls das Spiel zu einem Turnier gehört, soll diese Information ebenfalls gespeichert werden.
	\item Falls in einem Spiel Tore geschossen wurden, soll die Anzahl der Tore gespeichert werden.
	\item Ein Schiedsrichter verfügt über die gleichen Daten wie ein Spieler ausser dass er keine Spielposition hat. Dafür wird bei dem Schiedsrichter das Datum der Schiesrichterprüfung und die Berechtigungsklasse gespeichert.
	\item Jedes Turnier hat eine eindeutige Nummer, einen Namen, ein Beginn- und Enddatum und die beteiligten Mannschaften gespeichert.
\end{itemize}