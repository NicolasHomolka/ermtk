\section{Python}
\fib{}
\subsection{Entwicklung}

\noindent
Die im Jahre 1991 von Guido van Rossum, einem niederländischen Software-Entwickler, veröffentlichte Programmiersprache Python wurde ursprünglich für das Betriebssystem Amoeba entwickelt und sollte die Programmier-Lehrsprache ABC ablösen.
Mit der ersten Vollversion, die unter dem Namen Python 1.0 im Jahre 1994 erschienen ist, wurden einige Konzepte der funktionalen Programmierung implementiert, jedoch wurden diese später wieder aufgegeben. 
Im Jahre 2000 erschien Python 2.0 mit einer voll funktionsfähigen Garbage Collection, sowie die Unterstützung für den Unicode-Zeichensatz.
Python 3.0 wurde 8 Jahre später am 3.Dezember 2008 veröffentlicht. Mit der Version 3.0 kamen tiefgreifende Änderungen an der Sprache, die dazu führten, dass sich die Python Software Foundation dafür entschied, Python 2.7 und Python 3.0 bis Ende 2019 parallel mit neuen Versionen zu unterstützen. 
Die neuste Version ist Version 3.7, die am 27.Juni 2018 erschienen ist.
\footcite{noauthor_history_nodate}

\subsection{Idee und Zweck}

\noindent
Python ist eine sehr übersichtliche und einfache Programmiersprache und war in erster Linie dafür gedacht, das Programmieren zu erlernen. Das wird erzielt, in dem Python mit wenigen Schlüsselwörtern auskommt und die Syntax, im Vergleich zu anderen Programmiersprachen, sehr reduziert ist. Außerdem ist die Standardbibliothek von Python überschaubar und leicht erweiterbar, was zu folge hatte, dass heute eine Vielzahl an Bibliotheken zu Verfügung stehen. 
\footcite{noauthor_general_nodate}

\subsection{Verwendung}

\noindent
Da Python über eine große Vielfalt an Bibliotheken verfügt, einfach zu programmieren ist und der Code leicht zu lesen ist, bot es sich für dieses Projekt hervorragend an. Beispielsweise für die Implementierung der Umsetzung mittels Graphviz, die durch die von Graphviz zu Verfügung gestellten Bibliothek mit dem gleichen Namen programmiert wurde.
Ein anderes Beispiel wo dieses Projekt von Python profitiert hat, ist die Eigenschaft von Python, dass es möglich ist, Python-Programme als Module in anderen Sprachen einzubetten.
\footcite{noauthor_integrateddevelopmentenvironments_nodate}