\section{Rettungsstelle}
\fib{}

\noindent
Die Rettungsstelle soll den Aufbau einer echten Rettungsstelle darstellen und dabei veranschaulichen welche Vorgänge bzw. Ressourcen miteinander in Beziehung stehen.
Das Datenmodell der Rettungsstelle besteht aus sieben Entity-Typen:

\begin{itemize}
    \item Einsatz oder auch mission
    \item Fahrt oder auch trip
    \item Person oder auch person
    \item Angestellter oder auch employee
    \item Patient oder auch patient
    \item Auto oder auch car
    \item Gragage oder auch garage
\end{itemize}

\noindent
Zwischen diesen Entity-Typen existieren sechs Relationen und jede dieser Relationen kann man mit je zwei Sätzen beschreiben.
\begin{itemize}
    \item Ein Einsatz besteht aus keiner oder mehreren Fahrten.\newline
          Eine Fahrt ist Bestandteil genau eines Einsatzes.
    \item Ein Mitarbeiter nimmt an keiner oder mehreren Fahrten teil. \newline
          Bei einer Fahrt ist mindestens ein Mitarbeiter oder mehrere Mitarbeiter beteiligt. 
    \item Ein Patient wird bei genau einem Einsatz gerettet. \newline
          Bei einer Fahrt wird mindestens ein Patient oder mehrere Patienten gerettet. 
    \item Ein Auto wird bei mindestens einer Fahrt oder mehreren Fahrten verwendet. \newline
          Bei einer Fahrt wird genau ein Auto verwendet.
    \item Ein Auto hat mindestens eine Garage oder mehrere Garagen. \newline
          In einer Garage steht genau ein Auto.
    \item Patient ist eine Person mit einer SVN, KVA, Einsatzbeschreibung, Kennzeichen, Einsatznummer. \newline
          Angestellter ist eine Person mit einer Mittarbeiternummer, Rang, Schulung, Kennzeichen, Einsatznummer.
\end{itemize}

\noindent