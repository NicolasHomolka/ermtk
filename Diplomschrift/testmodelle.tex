\chapter{Testmodelle}
\label{cha:Testmodelle}
\fib{}
\section{Überblick}

\noindent
Bei dem Projekt gab es 9 Testmodelle die wären:
\begin{itemize}
    \item Fußball
    \item Kinokette
    \item Mondial
    \item Rettungsstelle
    \item Schulungsfirma
    \item Schulinformationssystem
    \item Tankstellenkette
    \item Weingut
\end{itemize}
\noindent

\noindent
Jedes dieser Testmodelle stellt eine Datenbank da. Alle 9 mussten in Form eine XERML-Datei erstellt werden um sie dann mit den vier verschiedenen Programmen zu verarbeiten. Darin sollten alle Variationen vorhanden sein z.B. Vererbungen, Abhängige-Typen usw..

\noindent
Diese XERML-Dateien können mit einem Tool eines anderen Projekts automatisch generiert werden.
Sie bestehn in der Regel aus drei verschiedenen Dateien.
\begin{itemize}
    \item Grunddatei mit der Endung "xerml.xml", in dieser Datei steht der grobe Aufbau des Modells.
    \item Sprachdatei wo verschiedene Übersetzungen enthalten sind mit der Endung "xerml.lo.xml".
    \item Typdatei die die einzelnen Attribute einer Entity beschreibt mit der Endung "xerml.ty.xml".
\end{itemize}